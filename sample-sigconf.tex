\documentclass[sigconf]{acmart}

\usepackage{booktabs} % For formal tables

\usepackage{etex}

% Copyright
%\setcopyright{none}
%\setcopyright{acmcopyright}
%\setcopyright{acmlicensed}
\setcopyright{rightsretained}
%\setcopyright{usgov}
%\setcopyright{usgovmixed}
%\setcopyright{cagov}
%\setcopyright{cagovmixed}


% DOI
\acmDOI{10.475/123_4}

% ISBN
\acmISBN{123-4567-24-567/08/06}

%Conference
\acmConference[WOODSTOCK'97]{ACM Woodstock conference}{July 1997}{El
  Paso, Texas USA} 
\acmYear{2017}
\copyrightyear{2017}

\acmPrice{370,000.00}


\begin{document}
\title{Polycommit: Forming Habits Through Gamification}
% \titlenote{Produces the permission block, and
%  copyright information}



\author{Elliot Fiske}
\affiliation{%
	\institution{California Polytechnic State University}
	%  \streetaddress{P.O. Box 1212}
	%\city{Dublin} 
	%\state{Ohio} 
	%\postcode{43017-6221}
}
\email{elliotfiske@gmail.com}

\author{Foaad Khosmood}
\affiliation{%
	\institution{California Polytechnic State University}
}
\email{foaad@calpoly.edu}


\begin{abstract}
\par Learning applications and platforms such as Polylearn are derided for being difficult to use and not engaging. With \textit{Commit,} we demonstrate that through gamification and by learning from user behavior, we can create educational software that improves retention of course content and improves test scores.

\par Using learning techniques such as spaced repitition and variable rewards scheduling, we created a learning platform that is engaging and compelling for users to use. We took favorite features from existing learning platforms such as Duolingo and Memrise, and refined the app's design based on feedback and data from users. Overall, we show that with polish and care shown to the user's motivations, we can create an engaging educational app that students will actually want to use, and enjoy using.

\par We developed \textit{Commit} for use with several Cal Poly classes. We worked directly with teachers to custom-make questions to be used throughout the app. We tested the app in 5 different classes, with around 50 active participats. We found that students who used the application performed 13\% better on tests and recalled 18\% more class knowledge 1 month after completing the course.
\end{abstract}

%
% The code below should be generated by the tool at
% http://dl.acm.org/ccs.cfm
% Please copy and paste the code instead of the example below. 
%
\begin{CCSXML}
<ccs2012>
 <concept>
  <concept_id>10010520.10010553.10010562</concept_id>
  <concept_desc>Computer systems organization~Embedded systems</concept_desc>
  <concept_significance>500</concept_significance>
 </concept>
 <concept>
  <concept_id>10010520.10010575.10010755</concept_id>
  <concept_desc>Computer systems organization~Redundancy</concept_desc>
  <concept_significance>300</concept_significance>
 </concept>
 <concept>
  <concept_id>10010520.10010553.10010554</concept_id>
  <concept_desc>Computer systems organization~Robotics</concept_desc>
  <concept_significance>100</concept_significance>
 </concept>
 <concept>
  <concept_id>10003033.10003083.10003095</concept_id>
  <concept_desc>Networks~Network reliability</concept_desc>
  <concept_significance>100</concept_significance>
 </concept>
</ccs2012>  
\end{CCSXML}

\ccsdesc[500]{Computer systems organization~Embedded systems}
\ccsdesc[300]{Computer systems organization~Redundancy}
\ccsdesc{Computer systems organization~Robotics}
\ccsdesc[100]{Networks~Network reliability}

% We no longer use \terms command
%\terms{Theory}

\keywords{Education, Gamification}

\maketitle

\input{samplebody-conf}

\bibliographystyle{ACM-Reference-Format}
\bibliography{sigproc} 

\end{document}
